\begin{abstract}
Mobile devices that we use, have a lot of in-built sensors that are capable of collecting data about the user. We also use our devices to access our contacts, photos, calendar, email, social networking sites and other internet or ad-hoc network based resources. As a result, a substantial amount of our personal information either is accessed and stored on these devices. Mobile operating systems expose APIs to applications(or apps) installed on the phone to access these stored data. The intention is to provide services to the user of the mobile device. Unfortunately, the current privacy and security mechanisms that control the data access use an ``all or nothing'' policy. Such a policy is inadequate in controlling data in specific contextual situations. There has been a lot of research in this area ranging from detection of data flow to prevention of data flow. These solutions achieve the required control by modifying the APIs or the mobile operating system itself. In this paper we present SPrivacy, a system built which uses privacy policies that define which app will be able to get access to the data, to what degree will this access be allowed and under what user context will such an access be allowed. We use SPrivacy to control ``suspect'' apps from accessing private information on the user's mobile device given the user context. We do this by static analysis and ``constants'' manipulation in the app's installer file. We make no changes to the mobile device's operating system or its APIs. We have implemented SPrivacy as per the definitions of the XACML standard for Attribute Based Access Control. We evaluate SPrivacy to show that it causes minimal overhead on the performance of the system.
\end{abstract}

% IEEEtran.cls defaults to using nonbold math in the Abstract.
% This preserves the distinction between vectors and scalars. However,
% if the conference you are submitting to favors bold math in the abstract,
% then you can use LaTeX's standard command \boldmath at the very start
% of the abstract to achieve this. Many IEEE journals/conferences frown on
% math in the abstract anyway.

% \begin{keywords)
% Mobile Applications, Access controls.
% \end{keywords}

% For peer review papers, you can put extra information on the cover
% page as needed:
% \ifCLASSOPTIONpeerreview
% \begin{center} \bfseries EDICS Category: 3-BBND \end{center}
% \fi
%
% For peerreview papers, this IEEEtran command inserts a page break and
% creates the second title. It will be ignored for other modes.
%c\IEEEpeerreviewmaketitle
